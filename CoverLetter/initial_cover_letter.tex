%% start of file `template.tex'.
%% Copyright 2006-2013 Xavier Danaux (xdanaux@gmail.com).
%
% This work may be distributed and/or modified under the
% conditions of the LaTeX Project Public License version 1.3c,
% available at http://www.latex-project.org/lppl/.

\documentclass[10pt,a4paper,roman]{moderncv}        % possible options include font size ('10pt', '11pt' and '12pt'), paper size ('a4paper', 'letterpaper', 'a5paper', 'legalpaper', 'executivepaper' and 'landscape') and font family ('sans' and 'roman')
\usepackage[english]{babel}


% moderncv themes
\moderncvstyle{classic}                            % style options are 'casual' (default), 'classic', 'oldstyle' and 'banking'
\moderncvcolor{green}                              % color options 'blue' (default), 'orange', 'green', 'red', 'purple', 'grey' and 'black'
%\renewcommand{\familydefault}{\sfdefault}         % to set the default font; use '\sfdefault' for the default sans serif font, '\rmdefault' for the default roman one, or any tex font name
%\nopagenumbers{}                                  % uncomment to suppress automatic page numbering for CVs longer than one page

\usepackage[document]{ragged2e}

% character encoding
\usepackage[utf8]{inputenc}                       % if you are not using xelatex ou lualatex, replace by the encoding you are using
%\usepackage{CJKutf8}                              % if you need to use CJK to typeset your resume in Chinese, Japanese or Korean

% adjust the page margins
\usepackage[scale=0.75]{geometry}
%\setlength{\hintscolumnwidth}{3cm}                % if you want to change the width of the column with the dates
%\setlength{\makecvtitlenamewidth}{10cm}           % for the 'classic' style, if you want to force the width allocated to your name and avoid line breaks. be careful though, the length is normally calculated to avoid any overlap with your personal info; use this at your own typographical risks...

% personal data
\name{FirstName}{LastName}
\title{Cover Letter}                               % optional, remove / comment the line if not wanted
\address{XXXX}% optional, remove / comment the line if not wanted; the "postcode city" and and "country" arguments can be omitted or provided empty
\phone[mobile]{XXXX}                   % optional, remove / comment the line if not wanted
\phone[fixed]{XXXX}                    % optional, remove / comment the line if not wanted
%\phone[fax]{+3~(456)~789~012}                      % optional, remove / comment the line if not wanted
\email{XXXX@XXXX.XXXX.XXXX}                               % optional, remove / comment the line if not wanted
\homepage{www.XXXX.com}                         % optional, remove / comment the line if not wanted
%\extrainfo{additional information}                 % optional, remove / comment the line if not wanted
%\photo[64pt][0.4pt]{picture}                       % optional, remove / comment the line if not wanted; '64pt' is the height the picture must be resized to, 0.4pt is the thickness of the frame around it (put it to 0pt for no frame) and 'picture' is the name of the picture file
%\quote{Some quote}                                 % optional, remove / comment the line if not wanted

% to show numerical labels in the bibliography (default is to show no labels); only useful if you make citations in your resume
%\makeatletter
%\renewcommand*{\bibliographyitemlabel}{\@biblabel{\arabic{enumiv}}}
%\makeatother
%\renewcommand*{\bibliographyitemlabel}{[\arabic{enumiv}]}% CONSIDER REPLACING THE ABOVE BY THIS

% bibliography with mutiple entries
%\usepackage{multibib}
%\newcites{book,misc}{{Books},{Others}}
%----------------------------------------------------------------------------------
%            content
%----------------------------------------------------------------------------------
\begin{document}
%-----       letter       ---------------------------------------------------------
% recipient data
\recipient{Dr. XXXX XXXX}{Editor-in-Chief, XXXX}
\date{\today}
\opening{Dear Dr. XXXX,}
\closing{Thank you for your consideration.\\\\
Sincerely,}
%\enclosure[Adjunto]{CV}          % use an optional argument to use a string other than "Enclosure", or redefine \enclname
\makelettertitle

\justify

I am pleased to submit an original research article entitled “XXXX” for consideration for publication in \textit{XXXX}. 

This is dummy copy. It is not meant to be read. It has been placed here solely to demonstrate the look and feel of finished, typeset text. Only for show. 

Far far away, behind the word mountains, far from the countries Vokalia and Consonantia, there live the blind texts. Separated they live in Bookmarksgrove right at the coast of the Semantics, a large language ocean. A small river named Duden flows by their place and supplies it with the necessary regelialia. It is a paradisematic country, in which roasted parts of sentences fly into your mouth. Even the all-powerful Pointing has no control about the blind texts it is an almost unorthographic life One day however a small line of blind text by the name of Lorem Ipsum decided to leave for the far World of Grammar. The Big Oxmox advised her not to do so, because there were thousands of bad Commas, wild Question Marks and devious Semikoli, but the Little Blind Text didn’t listen. She packed her seven versalia, put her initial into the belt and made herself on the way. When she reached the first hills of the Italic Mountains, she had a last view back on the skyline of her hometown Bookmarksgrove, the headline of Alphabet Village and the subline of her own road, the Line Lane. Pityful a rethoric question ran over her cheek, then




\vspace{0.5cm}


\makeletterclosing

\end{document}


%% end of file `template.tex'.
